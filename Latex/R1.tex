% ********** Rozdział 1 **********
\chapter{Opis założeń i cele projektu}
\section{Cel projektu}



Celem projektu jest stworzenie kompleksowego systemu informatycznego symulującego działanie automatu z napojami. Głównym celem jest usprawnienie zarządzania produktami, transakcjami oraz operacjami administracyjnymi w systemie. Projekt zakłada zautomatyzowanie kluczowych procesów, takich jak zakup produktów, obsługa płatności, zarządzanie stanem automatu oraz funkcje administracyjne, w tym zarządzanie asortymentem i przeglądanie transakcji. Realizacja projektu ma na celu poprawę efektywności operacji, zapewnienie intuicyjnej obsługi użytkownika oraz stworzenie stabilnego środowiska do zarządzania automatem vendingowym.



\section{Opis założeń}

Poniższe założenia mają na celu stworzenie funkcjonalnego systemu symulującego automat vendingowy, który umożliwi łatwe zarządzanie produktami, transakcjami oraz operacjami administracyjnymi, zapewniając jednocześnie przyjazną obsługę dla użytkownika.

\begin{itemize}
\item \textbf{Zarządzanie asortymentem}: System umożliwia pełne zarządzanie produktami dostępnymi w automacie z napojami. Administrator może dodawać nowe produkty, aktualizować ceny, modyfikować dostępne ilości oraz usuwać produkty, które nie są już potrzebne.

\item \textbf{Obsługa transakcji}: System zapewnia płynne przetwarzanie zakupów dokonywanych przez użytkowników. Pozwala na realizację transakcji w czasie rzeczywistym, obsługę płatności, wydawanie reszty oraz zapisywanie historii zakupów w bazie danych.

\item \textbf{Obsługa klienta}: Projekt zakłada intuicyjny interfejs użytkownika, który pozwala na łatwy wybór produktów, dokonywanie płatności oraz korzystanie z prostych komunikatów systemowych. 

\item \textbf{Panel administracyjny}: System oferuje dedykowany panel dla administratorów, który umożliwia przeglądanie transakcji, monitorowanie stanu automatu, zarządzanie środkami pieniężnymi w urządzeniu oraz dokonywanie wypłat gotówki z automatu.

\item \textbf{Bezpieczeńwość danych}: System uwzględnia odpowiednie mechanizmy zabezpieczeń, takie jak hasła dla panelu administratora oraz walidacja danych wejściowych, aby zapewnić ochronę przed błędami lub nadużyciami.

\item \textbf{Efektywność operacyjna}: Celem projektu jest stworzenie niezawodnego i łatwego w utrzymaniu systemu, który zautomatyzuje kluczowe procesy, zmniejszy ryzyko błędów i usprawni zarządzanie automatem vendingowym.

\item \textbf{Skalowalność}: Projekt został stworzony z myślą o łatwej rozbudowie i adaptacji, aby sprostać rosnącym wymaganiom i zmieniającym się potrzebom.

\end{itemize}







