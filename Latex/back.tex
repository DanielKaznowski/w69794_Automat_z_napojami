%spis rysunków
\addcontentsline{toc}{chapter}{Spis rysunków}
\listoffigures
\newpage


%streszczenie
\addcontentsline{toc}{chapter}{Streszczenie}
\noindent
{\footnotesize{}\textbf{Wyższa Szkoła Informatyki i Zarządzania z siedzibą w Rzeszowie\\
Kolegium Informatyki Stosowanej}
\vspace{30pt}

\begin{center}
\textbf{Streszczenie pracy dyplomowej inżynierskiej}\\
\temat
\end{center}

\vspace{30pt}
\noindent
\textbf{Autor: \autor
\\Promotor: \promotor
\\Słowa kluczowe: Automat z napojami, C\#, SQL Server }
\vspace{40pt}
\\Celem projektu jest stworzenie kompleksowego systemu informatycznego wspomagającego zarządzanie automatami z napojami, który umożliwia użytkownikom przeglądanie, wybieranie i zakup produktów dostępnych w maszynach. Projekt zakłada usprawnienie zarządzania stanem maszyn, ich asortymentem oraz procesami związanymi z obsługą klienta. Założenia projektu obejmują m.in. zarządzanie produktami, obsługę transakcji, bezpieczeństwo danych, generowanie raportów oraz import/eksport danych. Projekt ma na celu zautomatyzowanie wielu procesów, poprawę obsługi klienta oraz optymalizację kosztów operacyjnych. \newline
Projekt został zaimplementowany w języku C\# (.NET 7.0) przy użyciu narzędzi takich jak Visual Studio do programowania oraz SQL Server do przechowywania danych aplikacji. Minimalne wymagania sprzętowe dla uruchomienia aplikacji obejmują m.in. procesor 1 GHz, pamięć RAM 1-2 GB oraz system operacyjny Windows 7 lub nowszy. \newline
Struktura bazy danych składa się z kilku tabel, zapewniając efektywne przechowywanie i zarządzanie danymi związanych z funkcjonowaniem automatów vendingowych.
\vspace{80pt}

\noindent
\textbf{The University of Information Technology and Management in Rzeszow\\
Faculty of Applied Information Technology}
\vspace{30pt}

\begin{center}
\textbf{Thesis Summary\\}
Drink Vending Machine Application
\end{center}

\vspace{30pt}
\noindent
\textbf{Author: \autor
\\Supervisor: \promotor
\\Key words: Drink Vending Machine, C\#, SQL Server}
\vspace{40pt}
\\The aim of the project is to create a comprehensive IT system supporting the management of vending machines, which allows users to browse, select, and purchase products available in the machines. The project aims to improve the management of machine statuses, their inventory, and processes related to customer service. The project's assumptions include, among others, product management, transaction handling, data security, report generation, and data import/export. The project is designed to automate many processes, improve customer service, and optimize operational costs. \newline
The project was implemented in C\# (.NET 7.0) using tools such as Visual Studio for programming and SQL Server for storing application data. The minimum hardware requirements to run the application include a 1 GHz processor, 1-2 GB RAM, and a Windows 7 or newer operating system. \newline
The database structure consists of several tables, ensuring efficient storage and management of data related to the operation of the vending machines.
}
